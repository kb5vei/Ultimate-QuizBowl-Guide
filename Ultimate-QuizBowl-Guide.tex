
\documentclass[12pt]{book}
\usepackage{cclicenses}
\usepackage[letterpaper,  hmargin = { 1in},vmargin = { 1in}]{geometry}
\usepackage{hyperref}
\usepackage[T1]{fontenc}
\usepackage{multicol}


\begin{document}
	\frontmatter
	\title{Ultimate QuizBowl Guide}
	\author{Jonas Williamson and the  J. M. Hanks High School High-Q Team}
	\date{\the\year}
	\maketitle
	\begin{center}

			\huge \cc \the\year
			\vspace{0.5 in}
			
			\byncsa
	\normalsize
		\end{center}
		\vspace{2 in}	
	This work is licensed under the \href{https://creativecommons.org/licenses/by-nc-sa/4.0/legalcode}{Creative Commons Attribution-NonCommercial-ShareAlike 4.0 license}, some rights reserved.  You are allowed to copy, distribute and modify this work as long as the following conditions are met:
	\begin{enumerate}
		\item Attribution: The original author(s) of the work must be given proper credit.
		\item Non-commercial: This work cannot be used for commercial purposes.
		\item Share-Alike: If you modify this work, the work that is created must be shared under the same license. 
	\end{enumerate}


	
	
	
	
	
	
	\tableofcontents

	\mainmatter
	
	
\chapter{Introduction}
So you've discovered the world of trivia. QuizBowl, High-Q, or whatever name it is called in your area can be great fun to play, and sometimes even casual teams can win if they employ the right strategies and have a little bit of luck.  However, a team that wants to win consistently needs to take its preparation a little more seriously.  

There are 12 broad categories in quizbowl:
\begin{multicols}{2}
\begin{itemize}
	\item Current Events
	\item The Arts
	\item Geography
	\item History
	\item Literature
	\item Mythology
	\item Pop Culture
	\item Science
	\item Social Science
	\item Sports
	\item Theology and Philosophy
	\item Miscellaneous questions that don't belong in any of the other topics.
\end{itemize}
\end{multicols}
A general rule of thumb is that a team should consist of 4 players.  Rather than trying to be a generalist at first, each player should pick 3 topics to become an expert in.  While progress can be made by looking over lists and old questions, you can become a better player if you create your own lists.   Memorizing lists will generally allow you to answer questions at the ``giveaway'' point in the question, but building detailed lists will allow you to answer questions during the "power" phase of the question.  You can practice writing your own questions as well.  

When building a team, keep in mind that there are 4 players, and thus the players on your team should pick complementary areas.  For instance, if one player is interested in developing their ability to answer science questions, another player should work on history, while another learns about sports, and the final player studies literature.  Don't build teams out of groups of your school's ``best'' players.  Instead, build teams where the players' strengths work together synergisticly. 




 

\chapter{Current Events}
	\section{Introduction to Current Events}
		The very nature of current events makes it extremely hard to write a section of a book on this.  Studying for the current events means keeping up with the news.  Whether it be online news sites like \href{http://www.cnn.com}{CNN's website}, watching the news on TV or even reading the newspaper, you must do this every day to be well informed about current events.  
		
		This section will change quite frequently, and some of the current events section will undoubtedly be moved into the "history" section as time passes.  For reference, this version of the Ultimate QuizBowl Guide was released on \today. 
				
	\section{Business}
	\section{Politics}
	\section{Science}

\chapter{The Arts}
	\section{Visual}
		\subsection{Architecture}
		
			\subsubsection{Antoni Gaud\'{i}}
			\subsubsection{Frank Lloyd Wright}
			\subsubsection{Mies Van der Rohe}
			\subsubsection{Philip Johnson}
			\subsubsection{Eero Saarinen}
			\subsubsection{Richard Rogers}
			\subsubsection{Frank Gehry}
			\subsubsection{Norman Foster}
			\subsubsection{Renzo Piano}
			\subsubsection{Santiago Calatrava}
			\subsubsection{Zaha Hadid}
			\subsubsection{Oscar Niemeyer}
			\subsubsection{Rem Koolhas}
			\subsubsection{Jeanne Gang}
			\subsubsection{Shigeru Ban}
		\newpage	
		\subsection{Sculpture}
		\subsection{Painting}
		\subsection{Other}

	\section{Performing}
		\subsection{Plays}
		\subsection{Operas}
		\subsection{Music}

			
\chapter{Geography}
	\section{Continents}
	\subsection{Africa}
	\subsection{Antarctica}
	\subsection{Asia}
	\subsection{Australia and Oceania}
	\subsection{Europe}
	\subsection{North America}
	\subsection{South America}


	\section{Oceans}

		


\chapter{History}
	\section{Ancient History}
	\section{American History}
	\section{European History}
	\section{Religious History}
	\section{World History}
	
	
	
\chapter{Literature}
	\section{American Literature}
		\subsection{Fiction}
		\subsection{Nonfiction}
			\subsubsection{Silent Spring}
		\subsection{Poetry}

		
	\section{European Literature}
		\subsection{Fiction}
		\subsection{Poetry}
		\subsubsection{Shakespeare}
	\section{Religious Literature}
	\section{World Literature}
	
\chapter{Mythology}
	\section{Greek/Roman}
	\section{Egyptian}
	\section{Hawaiian/Polynesian}
	\section{Norse}
	\section{Mezoamerican}
	\section{Indian/South Asian}
	\section{Chinese/Japanese/East Asian}
	\section{Arthurian}



	

	

		
	
\chapter{Pop Culture}
	\section{Pop Music}
	\section{Entertainment}
		\subsection{Video Games}
		\subsection{Wrestling}



\chapter{Science}
	\section{Astronomy}
		\subsection{Astronomers}
			\subsubsection{Galileo}
			\subsubsection{Copernicus}
			\subsubsection{Kepler}
			\subsubsection{Cassini}
			\subsubsection{Edmund Halley}
			\subsubsection{Huyguens}
			\subsubsection{Clyde Tombaugh}
			\subsubsection{Kuiper}
			
		\subsection{Constellations}
			\subsubsection{Zodiac}
			\subsubsection{Other Constellations}
		
		
		\subsection{Cosmology}
		\subsection{Solar System}
			\subsubsection{Planets}
			\subsubsection{Dwarf Planets}
			\subsubsection{Sol}
			\subsubsection{Other Objects}
				\paragraph{Comets}
					\subparagraph{Halley's Comet}
					\subparagraph{Comet McNaught}
					\subparagraph{Hale-Bopp}
					\subparagraph{Shoemaker-Levi 9}
				\paragraph{Asteroids and Meteors}
				\paragraph{Interstellar Objects}
			
			\subsubsection{NASA}
				\paragraph{Manned Missions}
					\subparagraph{Mercury}
					\subparagraph{Gemini}
					\subparagraph{Apollo}
					\subparagraph{Skylab}
					\subparagraph{STS}
					\subparagraph{ISS}
					\subparagraph{SLS}
				\paragraph{Unmanned Missions}
				
			\subsubsection{Russian Space Programs}
			\subsubsection{Chinese Space Programs}
		\subsection{Stars}
		\subsection{Galaxies}
		\subsection{Telescopes}

	\section{Biology}
		\subsection{Microbiology}
			\subsubsection{Cells}
				\paragraph{Organelles}
		\subsection{Macrobiology}
			\subsubsection{Anatomy}
			\subsubsection{Taxonomy}
					
	\section{Chemistry}
	\section{Computer Science}
	\newpage
	\section{Earth Science}
		\subsection{Geology}
			\subsubsection{Geologic Time}
			\subsubsection{Extinction Level Events}
			\subsubsection{Vulcanology}
			\paragraph{Introduction to Vulcanology}
			\paragraph{Supervolcanoes}
			\paragraph{World Volcanoes}
			\newpage
			\paragraph{Hawaiian Volcanoes}
				\subparagraph{Introduction to the Hawaiian Volcanoes} - 

					The Hawaiian Volcanoes are due to a hot-spot underneath the pacific ocean which has formed the Emperor-Hawaiian Seamount Chain that stretches across the pacific to Japan and Siberia.  The volcanoes are on two parallel lines (the Loa Line and the Kea line - the Loa line is south (lower)).   

					The major islands of Hawaii are: 
					\begin{multicols}{2}
					\begin{itemize}
						\item Hawaii
						\item Maui
						\item Kaho'olawe
						\item Lanai
						\item Molokai
						\item Oahu
						\item Kauai
						\item Ni'ihau
					\end{itemize}
					\end{multicols}
					All of the Hawaiian islands were formed by volcanoes.  Only Maui and the Big Island have volcanoes that could still erupt.  

					\vspace{0.1in}

					\textbf{Volcanoes on the Big Island:}
					\begin{itemize}
						\item \textbf{Kilauea} - The youngest volcano.  Most active on earth.  Last erupted in 2018, devastating the lailani estates subdivision.  Traditional home the the Goddess Pele in Hawaiian mythology.  
						\item \textbf{Mauna Loa} - Largest volcano by mass in the world.  Second Most active volcano in Hawaii.  Last erupted in 1984, nearly destroying the city of Hilo (the largest city on the Big Island).  
						\item \textbf{Mauna Kea} - Tallest volcano in the pacific. (Remember, Mauna Loa is Lower in elevation).  Site involved in protests due to telescope contruction.  Jason Momoa (aquaman actor) Staged his arrest during the protest. 
						\item \textbf{Hualalai} - Smallest of the five volcanoes on the big island.  Known for xenoliths (rock from the mantle brought up in lava flows).  Last erupted in 1801.  Kona Airport is built on the 1801 lava flow.  
						\item \textbf{Kohala} - Oldest on the big island.  Experienced a Magnetic Field Reversal.  Experienced a huge landslide, fossils were deposited by a huge tsunami.  
					\end{itemize}
					\textbf{Other Hawaiian Volcanoes}
					\begin{itemize}
						\item Haleakal\=a - On the island of Maui. Still considered dormant.  Last erupted in 1790.  
						\item Lo'ih\=i  Is the newest of the Hawaiian volcanoes and is still underwater.  Will probably break the surface in 100,000 years or so.  
					\end{itemize}



					\subparagraph{Kilauea} - Things to know about Kilauea: 
						\begin{itemize}
							\item One of the most active Volcanoes on earth - often classified as the most active.
							\item Last Erupted 2018
							\item Semi-persistent Lava Lake at summit, disappeared in 2018.
							\item Erupted 1983-2018 at Pu'u 'O'o.
							\item Newest of the Hawaiian Volcanoes on the Big Island. (Lo'ih\=i is newer, but is still underwater.)
							\item Summit in Volcanoes National Park, near the town of Volcano.
							\item Traditional home of the goddess Pele.
							\item Located on the southeastern part of the Big Island on the Kea Line.
						\end{itemize}
						\textbf{2018 Eruption Facts: }
						\begin{itemize}
							\item Erupted in 2018 from fissures in the lower East Rift Zone. 
							\item Fissure 8 became dominant, decimating the Lelani Estates subdivision.
							\item Lava from the eruption created a new black sand beach. 
						\end{itemize}
						
						\textbf{Recent News: }
						\begin{itemize}
							\item A pool of water has formed where there used to be a lava lake. 
						\end{itemize}
						
						
						\textbf{Random facts:}
						\begin{itemize}
							\item Mark Twain once got lost while hiking into Kilauea's Caldera. 
							\item A man fell more that 70 feet into the caldera in May 2019.  He was rescued by helicopter. 
							\item Reading Rainbow filmed an episode on Kilauea.
							\item An eruption in 1790 killed at least 80 native Hawaiians. 
							\item Franklin D Roosevelt was the first president to visit Kilauea.
						\end{itemize}
				
					\newpage
					\subparagraph{Mauna Loa}
						Things to know about Mauna Loa:
						\begin{itemize}
							\item Mauna Loa is the largest volcano by mass in the world.  
							\item Mauna Loa is a very active volcano, second to Kilauea.
							\item There are atmospheric and Solar Observatories at the top of Mauna Loa.  The Atmospheric Observatory was responsible for the discovery of the Keeling curve for Carbon Dioxide.
							\item Mauna Loa is 13 679 ft tall, only 300 feet less that Mauna Kea.
							\item Mauna Loa last erupted in 1984.  The eruption nearly destroyed the town of Hilo.
							\item The summit and eastern flank of Mauna Loa are part of Hawaii Volcanoes National Park.  
						\end{itemize}
						
						\textbf{Recent News:} 
						\begin{itemize}
							\item Mauna Loa is currently (as of \today) on Yellow alert for volcanic eruption.
							\item 
						\end{itemize}
						
						\textbf{Random Facts}
						\begin{itemize}
							\item Mauna Loa is one of the 16 Decade volcanoes in the world chosen for monitoring because of their destructive history and proximity to population. 
							\item Coffee and Macadamian Nuts are grown on the slopes of Mauna Loa.
							
						\end{itemize}					
					
					\newpage
					\subparagraph{Mauna Kea}
						Things to know about Mauna Kea:
						\begin{itemize}
							\item Mauna Kea means "White Mountain" for the snow that often falls on its summit.
							\item It is the tallest volcano on Earth, and the highest peak in the Pacific, and the higest Island Mountain in the world.
							\item Mauna Kea is known for its numerous cinder cones.
							\item Also known as "Mauna a Waikea" meaning "The mountain of Waikea"
							\item The summit is sacred to Hawaiians, as it is where the Heavens and the Earth meet.
							\item Mauna Kea last erupted about 4600 years ago. 
	
							
						\end{itemize}
					\textbf{Recent News:}
							\begin{itemize}
								\item There are numerous telescopes near the summit of Mauna Kea, including Keck-1 and Keck-2.  Gerard Kuiper began the telescope program.  
								\item There have been ongoing protests to the building of the 30-meter telescope at the summit. 
								\item Actor Jason Momoa (Aquaman Actor) staged his arrest as part of the ongoing protests.
							\end{itemize}
					\textbf{Random Facts:}
							\begin{itemize}
								\item There is a glacial lake near the summit of Mauna Kea called Lake Waiau.
								\item Glacier-quenched Basalt can be found at the top of the mountain, indicating that in the last ice age, there was a glacier that covered the summit.  There is evidence of Early Hawaiians quarrying this harder, stronger, heavier rock. 
								\item Measured from its base on the ocean floor, it is the tallest mountain in the world.  Adding the sinking into the mantle of the pacific plate, it is nearly 70,000 feet tall, making Mauna Kea comparable to the Olympus Mons volcano on Mars (the largest volcano in the solar system).  
								\item The Mauna Kea Silversword is a plant that is only found on Mauna Kea (Another species of silversword is found on Haleakal\=a.)  In 2003 there were only 41 plants in the wild.  Conservation efforts have increased that number to nearly 8000, but the plant is still critically endangered. 
								\item Mauna Kea was the home of  Poli'ahu, deity of snow in Hawaiian mythology.
								\item The botanist David Douglas (for whom the Douglas Fir tree is named) died on Mauna Kea when he fell into a pit trap.  He may have been murdered.
								
								
							\end{itemize}


					\newpage
					\subparagraph{Hual\=alai}
						Things to know about Hual\=alai:
						\begin{itemize}
							\item Hual\=alai is the third youngest (and third oldest), and third most active of the five volcanoes on the Big Island.
							\item Hual\=alai last erupted in 1801.  Despite low levels of activity recently, it is still active and expected to erupt in the next century.
							\item Hual\=alai is the westernmost of the Big Island volcanoes.
							\item A major subfeature of Hual\=alai is Pu'u Wa'awa'a, Hawaiian for ``many-furrowed hill", a volcanic cone standing 372 m (1,220 ft) tall and measuring over 1.6 km (1 mi) in diameter.  The cone is made of Trachyte, a type of lava rock that exists no where else on the islands.
							\item The Kona Airport is built on a lava flow from Hual\=alai's 1801 eruption.
							\item Many resorts along the coast near Kona are built on historic Lava flows from Hual\=alai.
							
						\end{itemize}
					
					\subparagraph{Kohala} 
					Things to know about Kohala:
						\begin{itemize}
							\item Kohala is the oldest of the 5 volcanoes on the Big Island.
							\item Waipi'o Valley is a large eroded area in Kohala.
							\item It is old enough to have experienced a Magnetic field reversal that is recorded in its rocks about 780000 years ago.
							\item King Kamehameha I, the first King of the Kingdom of Hawaii, was born in North Kohala, near Hawi. 
						\end{itemize}
				
					\subparagraph{Hale'akala}
					\subparagraph{Lo'ih\=i}
		\subsection{Forensic Science}	
		\subsection{Meteorology}
		\subsection{Oceanography}
	\section{Physics}
		\subsection{Classical Physics}
		\subsection{Thermodynamics}
		\subsection{Modern Physics}
			\paragraph{Modern Physics Principles}
			\paragraph{Atomic and Nuclear Physics}
			\paragraph{Famous Modern Physics Experiments}


\chapter{Social Science}
	\section{Economics}
		\subsection{Historical Economics}
		\subsection{Economists}

	\section{Psychology}
	\section{Sociology}


\chapter{Sports}
	\section{Professional}
		\newpage
		\subsection{Baseball}
			\subsection{Teams}
			\begin{center}
				
				\begin{tabular}{|c|c|c|c|}
					\hline
					\textbf{Team} & \textbf{Location} & \textbf{Mascot} & \textbf{Venue} \\
					\hline
					Arizona Diamondbacks	& Phoenix & D. Baxter the Bobcat &Chase Field	\\
					\hline
					Atlanta Braves	& Cumberland & Blooper & SunTrust Park	\\
					\hline
					Baltimore Orioles 	& & &	\\
					\hline
					Boston Red Sox	& & &	\\
					\hline
					Chicago White Sox	& & &	\\
					\hline
					Chicago Cubs	& & &	\\
					\hline
					Cincinnati Reds	& & &	\\
					\hline
					Cleveland Indians	& & &	\\
					\hline
					Colorado Rockies	& & &	\\
					\hline
					Detroit Tigers	& & &	\\
					\hline
					Houston Astros	& & &	\\
					\hline
					Kansas City Royals	& & &	\\
					\hline
					Los Angeles Angels	& & &	\\
					\hline
					Los Angeles Dodgers	& & &	\\
					\hline
					Miami Marlins	& & &	\\
					\hline
					Milwaukee Brewers	& & &	\\
					\hline
					Minnesota Twins	& & &	\\
					\hline
					New York Yankees	& & &	\\
					\hline
					New York Mets	& & &	\\
					\hline
					Oakland Athletics	& & &	\\
					\hline
					Philadelphia Phillies	& & &	\\
					\hline
					Pittsburgh Pirates	& & &	\\
					\hline
					San Diego Padres	& & &	\\
					\hline
					San Francisco Giants	& & &	\\
					\hline
					Seattle Mariners	& & &	\\
					\hline
					St. Louis Cardinals	& & &	\\
					\hline
					Tampa Bay Rays	& & &	\\
					\hline
					Texas Rangers	& & &	\\
					\hline
					Toronto Blue Jays	& & &	\\
					\hline
					Washington Nationals	& & &	\\
					\hline
					
					
				\end{tabular}
			\end{center}
			
			
			
			
			
			\newpage
			\subsubsection{All Time Records as of \today}
			\begin{center}


				\begin{tabular}{|c|c|c|}
				\hline
					Highest batting average & Ty Cobb & .3664 \\
					\hline
					Most home runs & Barry Bonds & 762 \\
					\hline
					Most walks & Barry Bonds & 2,558 \\
					\hline
					Most grand slams & Alex Rodriguez &	25 \\
					\hline
					Most runs batted in & Hank Aaron  & 2,297 \\
					\hline
					Most hits & Pete Rose & 4,256 \\
					\hline
					Most singles & Pete Rose & 3,215 \\
					\hline
					Most at-bats & Pete Rose & 14,555 \\
					\hline
					Most games played & Pete Rose & 3,562 \\
					\hline 
					Most runs scored & Rickey Henderson & 2,295 \\
					\hline
					Most stolen bases & Rickey Henderson & 1,406 \\
					\hline
					Highest slugging percentage & Babe Ruth & .690 \\
					\hline
					Most strikeouts & Reggie Jackson & 2,597 \\
					\hline
					Most wins 	&	Cy Young 	&	511	\\
					\hline
					Most losses 	&	Cy Young 	&	316	\\
					\hline
					Most innings pitched 	&	Cy Young 	&	7,354$\frac{2}{3}$ 	\\
					\hline	
					Most complete games 	&	Cy Young 	&	749	\\				
					\hline
					Lowest E.R.A. 	&	Ed Walsh 	&	1.82	\\
					\hline
					Most no-hitters 	&	Nolan Ryan 	&	7	\\
					\hline
					Most strikeouts 	&	Nolan Ryan 	&	5714	\\
					\hline
					Most shutouts 	&	Walter Johnson 	&	110	\\
					\hline
					Most pickoffs 	&	Steve Carlton 	&	144	\\
					\hline
					Most hit batsmen 	&	Gus Weyhing 	&	278	\\
					\hline
					Most home runs allowed 	&	Jamie Moyer 	&	522	\\
					\hline
					Lowest walks plus hits per inning pitched 	&	Addie Joss 	&	0.968	\\
					\hline
					Most saves 	&	Mariano Rivera 	&	652	\\
					\hline
					Highest win–loss percentage 	&	Spud Chandler 	&	71.7\% 	\\
					\hline
					Most games 	&	Jesse Orosco 	&	1252	\\
					\hline
					Most consecutive scoreless innings pitched 	&	Orel Hershiser 	&	59\\
					\hline
					Most consecutive games played & Cal Ripken, Jr. & 2632 \\
					\hline
					Longest streak of games with a hit& Joe DiMaggio & 56 \\
					\hline
					Most hits in one season & Ichiro Suzuki& 262 \\
					\hline
					Most All-Star games played & Hank Aaron & 25 \\
					\hline
					Most World Series appearances (as a team)& New York Yankees &  40 \\
					\hline

				
					
				\end{tabular}
				\vspace{1 in}
			\end{center}			

			
			\newpage
		\subsection{Basketball}
			\subsubsection{All Time Records as of \today}
			\begin{center}
				\begin{tabular}{|c|c|c|}
					\hline
					Most Career Points Scored & Kareem Abdul-Jabbar & 38387 \\
					\hline
					 Most Career wins & Kareem Abdul-Jabbar & 1,074 \\
					\hline			
					Most Points Scored in a Single Season &  Wilt Chamberlain & 4,029 \\
					\hline
					Most Points Scored in a Single Game & Wilt Chamberlain & 100 \\
					\hline
					Most Career Rebounds & Wilt Chamberlain & 23974 \\
					\hline
					Most Career assists & John Stockton & 15806 \\
					\hline
					Most Career Steals & John Stockton & 3,265 \\
					\hline
					Most Career Blocks & Hakeem Olajuwon & 3830 \\
					\hline
					Highest points per game average & Michael Jordan & 30.12 \\
					\hline
				    Most points in a game without any fouls & Michael Jordan & 61 \\
				    \hline
				    Most free throws made & Karl Malone & 9,787 \\
				    \hline
			        Most starts & Karl Malone & 1,471 \\
			        \hline
			        Most games played & Robert Parish & 1,611 \\
			        \hline
			        Highest win percentage by a head coach& Steve Kerr & 78.5\% \\
			        \hline
		            Highest free throw percentage & Steve Nash & 90.43\%  \\
		            \hline
			        Most consecutive starts & Karl Malone & 1,395\\
			        \hline
		            Most dunks & Dwight Howard (Active) & 2,697 \\
			        \hline
		            Most consecutive games played & A.C. Green &  1,192 \\
		            \hline
		            Youngest player to be drafted & Andrew Bynum & 17 yrs and 249 days \\
		            \hline
		            Youngest NBA debut as a starter & LeBron James& 18 yrs and 303 days \\
		            \hline
	                Youngest player to start a game & Kobe Bryant & 18 yrs and 158 days \\
	                \hline	                
	                
	                
	                
		            
				      
				\end{tabular}	
			
			
			
			
			\end{center}		

			The Longest NBA game occurred on January 6, 1951 between the Olympians and Royals. Indianapolis beat Rochester 75–73 after 6 overtimes.
			
			
		
		\newpage		
		\subsection{Boxing}
		\subsection{Car Racing}
		\subsection{Curling}
		\subsection{Cycling}
		\subsection{Golf}
		\subsection{Gymnastics}
		\subsection{Hockey}
		\subsection{Figure Skating}
		\subsection{Football}
		\subsection{Skateboarding}		
		\subsection{Soccer}
		\subsection{Softball}
		\subsection{Swimming}
		\subsection{Tennis}
		\subsection{Track and Field}
		\subsection{Volleyball}
		\subsection{Weight Lifting}
		\newpage
	\section{College}
		\subsection{Baseball}
		\subsection{Basketball}
		\subsection{Golf}
		\subsection{Hockey}
		\subsection{Football}
		\subsection{Soccer}
		\subsection{Tennis}
	\newpage
	\section{Olympic Sports}
	
		\subsection{Summer Olympics}
			\subsubsection{Archery}
			\subsubsection{Badminton}
			\subsubsection{Baseball and Softball}
			\subsubsection{Basketball}
			\subsubsection{Beach Volleyball}
			\subsubsection{Boxing}
			\subsubsection{Canoe/Kayak}
			\subsubsection{Climbing}
			\subsubsection{Cycling}
				\paragraph{BMX}
				\paragraph{Mountain}
				\paragraph{Road}
				\paragraph{Track}
			\subsubsection{Diving}
			\subsubsection{Equestrian}
				\paragraph{Dressage}
				\paragraph{Jumping}
				\paragraph{Eventing}
			\subsubsection{Fencing}
			\subsubsection{Field Hockey}
			\subsubsection{Golf}
			\subsubsection{Gymnastics}
			\subsubsection{Handball}
			\subsubsection{Judo}
			\subsubsection{Karate}
			\subsubsection{Modern Pentathlon}
			\subsubsection{Roller Sport}
			\subsubsection{Rowing}
			\subsubsection{Rugby 7's}
			\subsubsection{Sailing}
			\subsubsection{Shooting}
			\subsubsection{Soccer}
			\subsubsection{Swimming}
			\subsubsection{Surfing}
			\subsubsection{Synchronized Swimming}
			\subsubsection{Table Tennis}
			\subsubsection{Taekwondo}
			\subsubsection{Tennis}
			\subsubsection{Track and Field}
			\subsubsection{Triathlon}
			\subsubsection{Volleyball (Indoor)}
			\subsubsection{Water Polo}
			\subsubsection{Weightlifting}
			\subsubsection{Wrestling}
			
			\newpage
		\subsection{Winter Olympics}
			\subsubsection{Alpine Skiing}
			\subsubsection{Biathalon}
			\subsubsection{Bobsleigh}
			\subsubsection{Cross Country Skiing}
			\subsubsection{Curling}
			\subsubsection{Figure Skating}
			\subsubsection{Freestyle Skiing}
			\subsubsection{Ice Hockey}
			\subsubsection{Luge}
			\subsubsection{Nordic Combined}
			\subsubsection{Short Track Skating}
			\subsubsection{Skeleton}
			\subsubsection{Ski Jumping}
			\subsubsection{Snowboarding}
			\subsubsection{Speed Skating}
		
		
\chapter{Theology and Philosophy}		
		
\chapter{Miscellaneous}







\end{document}