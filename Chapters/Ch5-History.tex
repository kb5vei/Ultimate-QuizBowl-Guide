\chapter{History}
	\section{Ancient History}
		
	\section{American History}
		\subsection{American Prehistory}
			\subsubsection{Geology of North America}
				\paragraph{Glaciers}
				\paragraph{Lake Agassiz}
				
				
				
			\subsubsection{Megafauna}
				\paragraph {Wooly Mamoth}
				\paragraph{Mastodon}
				\paragraph{Smilodon}
				\paragraph{Gylptodon}
				\paragraph{Giant Sloths}
				\paragraph{Dire Wolf}
		
		\subsection{Early Humans in North America}
			\subsubsection{Bearing Land Bridge}
			\subsubsection{Clovis culture}
			\subsubsection{Native American Tribes}
		\subsection{Colonial America}
		\subsection{American Revolution}
		\subsection{America under the Articles of Confederation}
		\subsection{From the Constitution to the War of 1812}
		\subsection{War of 1812}
		\subsection{From 1812 to the Civil War}
			\subsubsection{Texas Revolution}
		\subsection{Civil War} 
			\subsubsection{John Brown's Raid} 
			The raid inspired John Brown's Body, and later Battle Hymn of the Republic\footnote{See \cref{CivilWarSongs} on  \cpageref{CivilWarSongs}.}.
			\subsubsection{Secession and Ft. Sumter}
			\subsubsection{Battles}
			
		\subsection{Reconstruction}
		\subsection{From Reconstruction to Spanish American War}
		\subsection{Spanish American War}
		\subsection{From The Spanish American War to World War I}
		\subsection {World War I}
		\subsection{From World War 1 to the The Great Depression}
		\subsection{The Great Depression}
		\subsection{World War II}
		\subsubsection{The Cold War}

		\subsection{1990's to Present}
		
		
		
		
			
			
	
		\subsection{Presidents}
		\newpage
		\subsection{Supreme Court Cases}
			\subsubsection{Marbury v Madison}
			\subsubsection{McCulloch v Maryland}
			\subsubsection{Gibbons v Ogden}
			\subsubsection{Dred Scott v Sanford}
			\subsubsection{Plessy v Ferguson}
			\subsubsection{Schenck v United States}
			\subsubsection{Brown v Board of Education}
			\subsubsection{Miranda v Arizona}
			\subsubsection{Roe v Wade}
			\subsubsection{Texas v Johnson}
			\subsubsection{Obergefell v Hodges}
		\subsection{Congress}
		
	
		
		
	\section{European History}
	\section{Religious History}
		\subsection{Judaism}
		\subsection{Christianity}
			\subsubsection{Foundations}
			\newpage
			\subsubsection{Eccumenical Councils of the [Catholic] Church}
				\begin{tabular}{|c|c|c|}
					\hline
					\textbf{Council} & \textbf{Year (AD)} & \textbf{Notes} \\
					\hline
					Council of Jerusalem & about 50 & Mentioned in Acts 15 \\
					\hline
					Council of Nicea & 325 & Against Arianism; Date of Easter; Nicene Creed \\
					\hline
					Council of Constantinople & 381& Against Arianism and Pneumatomachi. \\
					\hline
					First Council of Ephesus & 431 & Against Nestorianism; Mary as Mother of God \\
					\hline
					Council of Chalcedon & 451 & Jesus was both true God and true man \\
					\hline 
					Second Council of Constantinople & 553&  Against Nestorianism\\
					\hline
					Third Council of Constantinople  & 680-681&  Against Monothelitism and monoenergism \\ 
					\hline
					Second Council of Nicaea & 787 & Against Iconoclasm; altars must contain relics.\\
					\hline
					Fourth Council of Constantinople & 869-870 & Condemned Photius and Iconoclasm \\
					\hline 
					First Council of the Lateran & 1123 & Investiture; Clerical Celibacy \\
					\hline
					Second Council of the Lateran & 1139 & Upheld first crusade declarations \\
					\hline
					Third Council of the Lateran & 1179 & Against Waldensian, Cathars.  \\
					\hline
					Fourth Council of the Lateran & 1215 & Transubstantiation; \\
					 & & Papal  Primacy; confession; 5th crusade. \\
					 \hline
					 First Council of Lyon & 1245 & Frederick II excommunicated; 7th crusade \\ 
					 \hline
					 Second Council of Lyon & 1274 & Dominican and Franciscan orders approved.\\
					 \hline
					 Council of Vienne & 1311-1312 & Knights Templar disbanded \\
					 \hline
					 Council of Constance & 1414-1418 & Ended Three-Popes Controversy; Conciliarism \\
					 \hline
					 Council of Basel,  & & \\ 
					 Ferrara and Florence &  1431-1445 & superiority of the Pope over the Councils \\
					 \hline
					Fifth Council of the Lateran &1512-1517 & purgatory; Mount of piety \\
					\hline
					Council of Trent & 1545-1563 & response to Protestantism; condemns \textit{sola fide}\\
					& & indulgences; 7 sacraments; biblical canon; \\
					& & this one is huge and really important \\
					\hline
					First Council of the Vatican & 1870&  never officially ended; Papal infallibility \\
					\hline
					Second Vatican Council & 1962-1965 & Mass in vernacular; lots of updates; 
					\\
					 & & This one is huge and really important too. \\
					 \hline
									 
					  
					
				\end{tabular}
				
			
			\newpage	
			\subsubsection{The Great Scism of 1054}
			\subsubsection{The Protestant Reformation}
				\paragraph{Martin Luther} \label{MartinLuther}
					Fun Fact: Martin Luther composed the hymn \textit{A Mighty Fortress is Our God}.  It is often sang on Reformation Sunday (The Last Sunday of October, or sometimes October 31).
				\paragraph{John Wesley}
				\paragraph{John Calvin}
				\paragraph{Henry VIII}
				
				
		\subsection{Islam}
		\subsection{Buddhism}
		\subsection{Hinduism}
		\subsection{}
	\section{World History}
	