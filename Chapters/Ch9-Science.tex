\chapter{Science}
	\section{Astronomy}
		\subsection{Astronomers}
			\subsubsection{Galileo}
			\subsubsection{Copernicus}
			\subsubsection{Kepler}
			\subsubsection{Cassini}
			\subsubsection{Edmund Halley}
			\subsubsection{Huyguens}
			\subsubsection{Clyde Tombaugh}
			\subsubsection{Kuiper}
			
		\subsection{Constellations}
			\subsubsection{Zodiac}
			\subsubsection{Other Constellations}
		
		
		\subsection{Cosmology}
		\subsection{Solar System}
			\subsubsection{Planets}
				\paragraph{Mercury}
				\paragraph{Venus}
				\paragraph{Earth}
				\paragraph{Mars}
				\paragraph{Jupiter} \label{Jupiter}
				\paragraph{Saturn}
				\paragraph{Uranus}
				\paragraph{Neptune}
				
			\subsubsection{Dwarf Planets}
				\paragraph{Pluto}
				\paragraph{Eris}
				\paragraph{MakeMake}
				\paragraph{Haumea}
				\paragraph{Ceres}
				
			\subsubsection{Sol}
				\paragraph{Age, Compostion and Size} 
				Our Sun is a middle-aged G-Class yellow dwarf\footnote{See \textbf{Spectral Class} on \cpageref{spectralclass}.}, and is approximately 4.6 billion years old.  It is composed of mostly Hydrogen (approximately 75\%) and Helium (approximately 25\%).  Other elements make up less that 0.1\% of the sun's mass.  
				
				It makes up 99.9\% of the solar system's mass (with the majority of the remaining mass in Jupiter\footnote{See \textbf{Jupiter} on page \cpageref{Jupiter}.}
				\paragraph{Sunspots}
				\paragraph{Prominences, Solar Flares, and Coronal Mass Ejections}
				\paragraph{Solar Wind and Solar Flux}
				
				
			\subsubsection{Other Objects}
				\paragraph{Comets}
					\subparagraph{Halley's Comet}
					\subparagraph{Comet McNaught}
					\subparagraph{Hale-Bopp}
					\subparagraph{Shoemaker-Levi 9}
				\paragraph{Asteroids and Meteors}
				\paragraph{Dwarf Planet Candidates}
					\subparagraph{Sedna}
					
				\paragraph{Interstellar Objects}
				
			\subsubsection{Space Firsts}
				You should learn the following important firsts in space:
				
			\begin{longtable}{|c|c|c|c|}
				\hline
				\textbf{Description} & {Object} & Country & Date \\
				\hline
				First Human-Made Object in Space & Sputnik 1 (Satellite) & USSR & ? \\
				\hline
				First Animal in Space & Laika (Dog) & USSR & ? \\
				\hline
				First Human in Space & Yuri Gagarin &  USSR & ? \\
				\hline
				First American in Space & Alan Shepard & USA & ? \\
				\hline
				First Woman in Space & Valentina Tershekova & USSR & ? \\
				\hline
				First American Woman in Space & Sally Ride & USA & ? \\
				\hline
				First Taikonaut & Yang Liwei & China & 2003-10-15\\
				\hline
			\end{longtable}
			
			\subsubsection{NASA}
				\paragraph{Manned Missions}
					\subparagraph{Mercury}
						
					\subparagraph{Gemini}
					\subparagraph{Apollo}
					\subparagraph{Skylab}
					\subparagraph{STS}
					\subparagraph{ISS}
					\subparagraph{SLS}
				\paragraph{Unmanned Missions}
					\subparagraph{Explorer}
						\textbf{Explorer 1} was the name of the first satellite launched by the USA.  It was the response to Sputnik (see \cpageref{sputnik}), but was significantly more advanced, with sensors that discovered the Van Allen Radiation Belts around the Earth. 
						
						Over 80 missions have been launched with the name Explorer.  Most have a satellite and are science based. 
				
					\subparagraph{Pioneer}
					\subparagraph{Echo}
					\subparagraph{Ranger}
					\subparagraph{Mariner}
					\subparagraph{Viking}
					\subparagraph{Voyager}
					\subparagraph{Galileo}
					\subparagraph{Ulyssees}
					\subparagraph{Discovery}
					\subparagraph{Cassini-Huygens}
					\subparagraph{Messenger}
					\subparagraph{New Horizons}
					\subparagraph{Juno}
					\subparagraph{Dawn}
					
					
					
			\subsubsection{Russian Space Programs}
				\paragraph {Manned}
				\paragraph {Unmanned}
					\subparagraph{Sputnik} \label{sputnik}
				
			\subsubsection{Chinese Space Programs}
		\subsection{Stars}
			\subsubsection{Types}
				\paragraph{Hertzsprung-Russel Diagrams}
				\paragraph{Spectral Class} \label{spectralclass}
		\subsection{Galaxies}
		\subsection{Telescopes}

	\section{Biology}
		\subsection{Microbiology}
			\subsubsection{Cells}
				\paragraph{Organelles}
		\subsection{Macrobiology}
			\subsubsection{Anatomy}
			\subsubsection{Taxonomy}
					
	\section{Chemistry}
	\section{Computer Science}
	\newpage
	\section{Earth Science}
		\subsection{Geology}
			\subsubsection{Geologic Time}
			\subsubsection{Extinction Level Events}
			\subsubsection{Vulcanology}
			\paragraph{Introduction to Vulcanology}
			\paragraph{Supervolcanoes}
			\paragraph{World Volcanoes}
				\subparagraph{Krakatoa}
				\label{Krakatoa} See The Scream by \ref{EdwardMunchArt} on \cpageref{EdwardMunchArt} 
				\subparagraph{Mt. Pinatubo}
				\subparagraph{Mt. Etna}
				\subparagraph{Kilimanjaro}
				\subparagraph{Mt. Erebus} 
				
			\newpage
			\paragraph{Hawaiian Volcanoes}
				\subparagraph{Introduction to the Hawaiian Volcanoes} - 

					The Hawaiian Volcanoes are due to a hot-spot underneath the pacific ocean which has formed the Emperor-Hawaiian Seamount Chain that stretches across the pacific to Japan and Siberia.  The volcanoes are on two parallel lines (the Loa Line and the Kea line - the Loa line is south (lower)).   

					The major islands of Hawaii are: 
					\begin{multicols}{2}
					\begin{itemize}
						\item Hawaii
						\item Maui
						\item Kaho'olawe
						\item Lanai
						\item Molokai
						\item Oahu
						\item Kauai
						\item Ni'ihau
					\end{itemize}
					\end{multicols}
					All of the Hawaiian islands were formed by volcanoes.  Only Maui and the Big Island have volcanoes that could still erupt.  

					\vspace{0.1in}

					\textbf{Volcanoes on the Big Island:}
					\begin{itemize}
						\item \textbf{Kilauea} - The youngest volcano.  Most active on earth.  Last erupted in 2018, devastating the lailani estates subdivision.  Traditional home the the Goddess Pele in Hawaiian mythology.  
						\item \textbf{Mauna Loa} - Largest volcano by mass in the world.  Second Most active volcano in Hawaii.  Last erupted in 1984, nearly destroying the city of Hilo (the largest city on the Big Island).  
						\item \textbf{Mauna Kea} - Tallest volcano in the pacific. (Remember, Mauna Loa is Lower in elevation).  Site involved in protests due to telescope contruction.  Jason Momoa (aquaman actor) Staged his arrest during the protest. 
						\item \textbf{Hualalai} - Smallest of the five volcanoes on the big island.  Known for xenoliths (rock from the mantle brought up in lava flows).  Last erupted in 1801.  Kona Airport is built on the 1801 lava flow.  
						\item \textbf{Kohala} - Oldest on the big island.  Experienced a Magnetic Field Reversal.  Experienced a huge landslide, fossils were deposited by a huge tsunami.  
					\end{itemize}
					\textbf{Other Hawaiian Volcanoes}
					\begin{itemize}
						\item Haleakal\=a - On the island of Maui. Still considered dormant.  Last erupted in 1790.  
						\item Lo'ih\=i  Is the newest of the Hawaiian volcanoes and is still underwater.  Will probably break the surface in 100,000 years or so.  
					\end{itemize}



					\subparagraph{Kilauea} \label{Kilauea}- Things to know about Kilauea: 
						\begin{itemize}
							\item One of the most active Volcanoes on earth - often classified as the most active.
							\item Last Erupted 2018; New eruption began on December 20, 2020.
							\item Semi-persistent Lava Lake at summit, disappeared in 2018, reappeared on December 20, 2020.
							\item Erupted 1983-2018 at Pu`u 'O'o.
							\item Newest of the Hawaiian Volcanoes on the Big Island. (Lo'ih\=i is newer, but is still underwater.)
							\item Summit in Volcanoes National Park, near the town of Volcano.
							\item Traditional home of the goddess Pele.
							\item Located on the southeastern part of the Big Island on the Kea Line.
						\end{itemize}
						\textbf{2018 Eruption Facts: }
						\begin{itemize}
							\item Erupted in 2018 from fissures in the lower East Rift Zone. 
							\item Fissure 8 became dominant, decimating the Lelani Estates subdivision.
							\item Lava from the eruption created a new black sand beach. 
						\end{itemize}
						
						\textbf{Recent News: }
						\begin{itemize}
							\item A pool of water has formed where there used to be a lava lake. 
						\end{itemize}
						
						
						\textbf{Random facts:}
						\begin{itemize}
							\item Mark Twain once got lost while hiking into Kilauea's Caldera. 
							\item A man fell more that 70 feet into the caldera in May 2019.  He was rescued by helicopter. 
							\item Reading Rainbow filmed an episode on Kilauea.
							\item An eruption in 1790 killed at least 80 native Hawaiians. 
							\item Franklin D Roosevelt was the first president to visit Kilauea.
						\end{itemize}
				
					\newpage
					\subparagraph{Mauna Loa}
						Things to know about Mauna Loa:
						\begin{itemize}
							\item Mauna Loa is the largest volcano by mass in the world.  
							\item Mauna Loa is a very active volcano, second to Kilauea.
							\item There are atmospheric and Solar Observatories at the top of Mauna Loa.  The Atmospheric Observatory was responsible for the discovery of the Keeling curve for Carbon Dioxide.
							\item Mauna Loa is 13 679 ft tall, only 300 feet less that Mauna Kea.
							\item Mauna Loa last erupted in 1984.  The eruption nearly destroyed the town of Hilo.
							\item The summit and eastern flank of Mauna Loa are part of Hawaii Volcanoes National Park.  
						\end{itemize}
						
						\textbf{Recent News:} 
						\begin{itemize}
							\item Mauna Loa is currently (as of \today) on Yellow alert for volcanic eruption.
							\item 
						\end{itemize}
						
						\textbf{Random Facts}
						\begin{itemize}
							\item Mauna Loa is one of the 16 Decade volcanoes in the world chosen for monitoring because of their destructive history and proximity to population. 
							\item Coffee and Macadamian Nuts are grown on the slopes of Mauna Loa.
							
						\end{itemize}					
					
					\newpage
					\subparagraph{Mauna Kea}
						Things to know about Mauna Kea:
						\begin{itemize}
							\item Mauna Kea means "White Mountain" for the snow that often falls on its summit.
							\item It is the tallest volcano on Earth, and the highest peak in the Pacific, and the higest Island Mountain in the world.
							\item Mauna Kea is known for its numerous cinder cones.
							\item Also known as "Mauna a Waikea" meaning "The mountain of Waikea"
							\item The summit is sacred to Hawaiians, as it is where the Heavens and the Earth meet.
							\item Mauna Kea last erupted about 4600 years ago. 
	
							
						\end{itemize}
					\textbf{Recent News:}
							\begin{itemize}
								\item There are numerous telescopes near the summit of Mauna Kea, including Keck-1 and Keck-2.  Gerard Kuiper began the telescope program.  
								\item There have been ongoing protests to the building of the 30-meter telescope at the summit. 
								\item Actor Jason Momoa (Aquaman Actor) staged his arrest as part of the ongoing protests.
							\end{itemize}
					\textbf{Random Facts:}
							\begin{itemize}
								\item There is a glacial lake near the summit of Mauna Kea called Lake Waiau.
								\item Glacier-quenched Basalt can be found at the top of the mountain, indicating that in the last ice age, there was a glacier that covered the summit.  There is evidence of Early Hawaiians quarrying this harder, stronger, heavier rock. 
								\item Measured from its base on the ocean floor, it is the tallest mountain in the world.  Adding the sinking into the mantle of the pacific plate, it is nearly 70,000 feet tall, making Mauna Kea comparable to the Olympus Mons volcano on Mars (the largest volcano in the solar system).  
								\item The Mauna Kea Silversword is a plant that is only found on Mauna Kea (Another species of silversword is found on Haleakal\=a.)  In 2003 there were only 41 plants in the wild.  Conservation efforts have increased that number to nearly 8000, but the plant is still critically endangered. 
								\item Mauna Kea was the home of  Poli'ahu, deity of snow in Hawaiian mythology.
								\item The botanist David Douglas (for whom the Douglas Fir tree is named) died on Mauna Kea when he fell into a pit trap.  He may have been murdered.
								
								
							\end{itemize}


					\newpage
					\subparagraph{Hual\=alai}
						Things to know about Hual\=alai:
						\begin{itemize}
							\item Hual\=alai is the third youngest (and third oldest), and third most active of the five volcanoes on the Big Island.
							\item Hual\=alai last erupted in 1801.  Despite low levels of activity recently, it is still active and expected to erupt in the next century.
							\item Hual\=alai is the westernmost of the Big Island volcanoes.
							\item A major subfeature of Hual\=alai is Pu'u Wa'awa'a, Hawaiian for ``many-furrowed hill", a volcanic cone standing 372 m (1,220 ft) tall and measuring over 1.6 km (1 mi) in diameter.  The cone is made of Trachyte, a type of lava rock that exists no where else on the islands.
							\item The Kona Airport is built on a lava flow from Hual\=alai's 1801 eruption.
							\item Many resorts along the coast near Kona are built on historic Lava flows from Hual\=alai.
							
						\end{itemize}
					
					\subparagraph{Kohala} 
					Things to know about Kohala:
						\begin{itemize}
							\item Kohala is the oldest of the 5 volcanoes on the Big Island.
							\item Waipi'o Valley is a large eroded area in Kohala.
							\item It is old enough to have experienced a Magnetic field reversal that is recorded in its rocks about 780000 years ago.
							\item King Kamehameha I, the first King of the Kingdom of Hawaii, was born in North Kohala, near Hawi. 
						\end{itemize}
				
					\subparagraph{Hale'akala}
					\subparagraph{Lo'ih\=i}
		\subsection{Forensic Science}	
		\subsection{Meteorology}
		\subsection{Oceanography}
	\section{Physics}
		\subsection{Classical Physics}
		\subsection{Thermodynamics}
		\subsection{Modern Physics}
			\paragraph{Modern Physics Principles}
			\paragraph{Atomic and Nuclear Physics}
			\paragraph{Famous Modern Physics Experiments}

