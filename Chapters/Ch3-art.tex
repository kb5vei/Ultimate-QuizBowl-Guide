\chapter{The Arts}
	\section{Visual}
		\subsection{Architecture}
		
			\subsubsection{Antoni Gaud\'{i}}
			\subsubsection{Frank Lloyd Wright}
			\subsubsection{Mies Van der Rohe}
			\subsubsection{Philip Johnson}
			\subsubsection{Eero Saarinen}
			\subsubsection{Richard Rogers}
			\subsubsection{Frank Gehry}
			\subsubsection{Norman Foster}
			\subsubsection{Renzo Piano}
			\subsubsection{Santiago Calatrava}
			\subsubsection{Zaha Hadid}
			\subsubsection{Oscar Niemeyer}
			\subsubsection{Rem Koolhas}
			\subsubsection{Jeanne Gang}
			\subsubsection{Shigeru Ban}
		\newpage	
		\subsection{Sculpture}
			\subsubsection{Praxiteles}
			\subsubsection{Donatello}
			\subsubsection{Michelangelo}
			\subsubsection{Gianlorenzo Bernini}
			\subsubsection{Auguste Rodin}
			\subsubsection{Constantin Brancusi}
			\subsubsection{Alberto Giacometti}
			\subsubsection{Henry Moore}
			\subsubsection{Sol LeWitt}
			\subsubsection{Louise Bourgeois}
			
		\subsection{Painting}
			\subsubsection{Marciel Duchamp}
			\subsubsection{Edward Hopper}
			\subsubsection{Andy Warhol}
			\subsubsection{Salvador Dali}
			\subsubsection{Pablo Picasso}
			\subsubsection{Vincent VanGogh}
			\subsubsection{Sandro Botticelli}
			\subsubsection{Monet}
			\subsubsection{Georges Sevrat}
			\subsubsection{Rembrant}
			\subsubsection{Francisco Goya}
			\subsubsection{El Greco}
			\subsubsection{Grant Wood}
			\subsubsection{Caravaggio}
			\subsubsection{Hokusai}
			\subsubsection{Eugene Delacroix}
			\subsubsection{William Blake}
			\label{WilliamBlakeArt} This Section is about William Blake's Artwork.  To see his writings, see \ref{WilliamBlakeLit} on \cpageref{WilliamBlakeLit}
			
			
		\subsection{Other}

	\section{Performing}
		\subsection{Plays}
			\subsubsection{Greek}
			\subsubsection{Shakespeare} \label{Shakespeare-Play} For Shakespere's poetry, see \ref{Shakespeare-Poetry} on page \cpageref{Shakespeare-Poetry}
			\subsubsection{Modern Plays}
			\subsubsection{Arthur Miller}
			
			
		\subsection{Operas}
		
		
		
		\paragraph{Gilbert and Sullivan}
			\subparagraph{HMS Pinafore}
			\subparagraph{The Pirates of Penzance}
			\subparagraph{The Mikado}
\newpage
		\subsection{Musicals}
		\paragraph{Leonard Bernstein} is the composer of West Side Story.  
		



		\paragraph {Frederick Loewe} composed my fair lady.
		
		My Fair Lady

		\paragraph{Cole Porter}
		Best known for Kiss Me Kate
			\begin{longtable}{|c|c|p{3in}|}
	
	
				\hline
				Kiss Me, Kate & 1948 & People are making a production of the Taming of the Shrew.  \\
				\hline
	
	
	
	
\end{longtable}
		\paragraph{Rogers and Hammerstein} test

			\begin{longtable}{|c|c|p{3in}|}
				\hline
				The Sound of Music & Year & Insert synopsis here.  Does this word-wrap if it is too long?  \\
				\hline
				Oklahoma! & Year & Insert synopsis here.  Does this word-wrap if it is too long?  \\
				\hline
				South Pacific & Year & Insert synopsis here.  Does this word-wrap if it is too long? \\
				\hline
				The King and I & Year & Insert synopsis here.  Does this word-wrap if it is too long?  \\
				\hline
				Carousel & Year & Insert synopsis here.  Does this word-wrap if it is too long? \\
				\hline
			\end{longtable}


		\paragraph{Andrew LLoyd Webber}
			Andrew Lloyd Webber composed 20 musicals, as well as a Requiem mass in Latin, and many other Stand-Alone Songs.  His best known musicals are listed below, along with a short synopsis:
			
				\begin{longtable}{|c|c|p{3in}|}
					\hline
					Joseph and the Amazing Technicolor Dreamcoat & 1968 & Insert synopsis here.  Does this word-wrap if it is too long?  \\
					\hline
					Jesus Christ, Superstar & 1970 & Insert synopsis here.  Does this word-wrap if it is too long?  \\
					\hline
					Evita & 1976 & Insert synopsis here.  Does this word-wrap if it is too long? \\
					\hline
					Cats & 1981 & Insert synopsis here.  Does this word-wrap if it is too long?  \\
					\hline
					Phantom of the Opera & 1986 & Insert synopsis here.  Does this word-wrap if it is too long? \\
					\hline
					
				
					
					
				\end{longtable}

		
		

		
	
		
	
		
		
		\subsection{Music}
			\subsection{Religious}
				\paragraph{Chant}
					\subparagraph{Te Deum}
					\subparagraph{Let All Mortal Flesh Keep Silence}
					\subparagraph{Phos Hilaron (Lumen Hilare)}
					\subparagraph{Pange, lingua} - St. Thomas Aquinas
					\subparagraph{Dies Irae}
				\paragraph{Religious Hymns}
					\subparagraph{A Mighty Fortress is our God} - Composed by Martin Luther\footnote{see \cref{MartinLuther} \cpageref{MartinLuther}}
					\subparagraph{Amazing Grace}
					\subparagraph{How Great Thou Art}


					
					
			\subsubsection{Classical}
				\paragraph{Composers}
					\subparagraph{Beethoven}
					\subparagraph{Mozart}
					\subparagraph{Bach}
					\subparagraph{Brahms}
					\subparagraph{Holst}
					\subparagraph{Hayden}
					\subparagraph{Chopin}
					\subparagraph{Dvorak}
					\subparagraph{Handel}
					\subparagraph{Tchaikovsky}
					\subparagraph{Aaron Copland}
					
				
				\paragraph{Themes Contained in Songs}
				
					\subparagraph {Beethoven's 9th} -  Ode to Joy
					\subparagraph{Aaron Copeland's Appalacian Spring} - Simple Gifts
				
			
			\newpage
			\subsubsection{Songs of the Civil War}
			\label{CivilWarSongs}
				\paragraph{When Johnny Comes Marching Home}
				\paragraph{John Brown's Body}
				\paragraph {Follow the Drinking Gourd}
				\paragraph{Lincoln and Liberty Too}
				\paragraph{Bonny Blue Flag}
				\paragraph{Battle Cry of Freedom}
				\paragraph{Goober Peas}
				\paragraph{Marching Through Georgia}
				\paragraph{Battle Hymn of the Republic} was written by Julia Ward Howe.  She fell asleep with the lyrics for John Brown's Body in her head.  She Awoke the next morning and wrote the new version in the morning twilight.\footnote{See Howe, Julia Ward. Reminiscences: 1819-1899. Houghton, Mifflin: New York, 1899. p. 275.; Quote available on \href{https://en.wikipedia.org/wiki/Battle_Hymn_of_the_Republic\#Creation_of_the_"Battle_Hymn"}{Wikipedia}}
				
			\subsubsection{Ragtime}
				\paragraph{The Entertainer} - Scott Joplin
				\paragraph{Maple Leaf Rag} - Scott Joplin
				
			\subsubsection{Jazz}
				\paragraph{Rhapsody in Blue} - George Gershwin
			
			\subsubsection{Rock}
			
			\subsubsection{Techno}
				
			