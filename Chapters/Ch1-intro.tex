\chapter{Introduction}
So you've discovered the world of trivia. QuizBowl, High-Q, or whatever name it is called in your area can be great fun to play, and sometimes even casual teams can win if they employ the right strategies and have a little bit of luck.  However, a team that wants to win consistently needs to take its preparation a little more seriously.  

There are 12 broad categories in quizbowl:
\begin{multicols}{2}
\begin{itemize}
	\item Current Events
	\item The Arts
	\item Geography
	\item History
	\item Literature
	\item Mythology
	\item Pop Culture
	\item Science
	\item Social Science
	\item Sports
	\item Theology and Philosophy
	\item Miscellaneous
\end{itemize}
\end{multicols}
A general rule of thumb is that a team should consist of 4 players.  Rather than trying to be a generalist at first, each player should pick 3 topics to become an expert in.  While progress can be made by looking over lists and old questions, you can become a better player if you create your own lists.   Memorizing lists will generally allow you to answer questions at the ``giveaway'' point in the question, but building detailed lists will allow you to answer questions during the "power" phase of the question.  You can practice writing your own questions as well.  

When building a team, keep in mind that there are 4 players, and thus the players on your team should pick complementary areas.  For instance, if one player is interested in developing their ability to answer science questions, another player should work on history, while another learns about sports, and the final player studies literature.  Don't build teams out of groups of your school's ``best'' players.  Instead, build teams where the players' strengths work together synergisticly. 
