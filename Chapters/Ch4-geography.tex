\chapter{Geography}
	\section{General World Geography}
		\begin{longtable}{|c|c|p{3in}|}
			\hline
			Highest Peak & Mt. Everest & On the Border between Nepal and China;  \\
			\hline
			Furthest from Center of Earth & Chimborazo & Ecuador \\
			\hline
			Greatest Vertical Drop & Mt. Thor & Barrin Island, CA, 4,101 feet \\
			\hline
			Lowest point on land & Dead Sea& Israel \\
			\hline
			Lowest point in the Ocean & Challenger Deep & In the Mariana Trench \\
			\hline
			Tallest Waterfall & Angel Falls & Venezuela \\
			\hline
			Longest River & Nile & Africa \\
			\hline
			Largest River by Volume & Amazon & South America \\
			\hline
			Deepest River & Congo & Africa \\
			\hline
			Largest Lake & Caspian Sea & Europe \\
			\hline
			Deepest Lake & Lake Baikal & Russia \\
			\hline
			Largest Continent & Asia & \\
			\hline
			Smallest Continent & Australia & \\ 
			\hline
			Largest Ocean & Pacific & \\
			\hline
			Smallest Ocean & Arctic & \\
			\hline
			
			
			
		\end{longtable}
	
	
	\section{Continents}
	\subsection{Africa}
		\subsubsection{Random Trivia}
		\begin{itemize} 
			\item Africa is the only continent in all four Hemispheres (Northern, Southern, Eastern, Western)
		\end{itemize}
	\subsection{Antarctica}
	\subsection{Asia}
	\subsection{Australia and Oceania}
	\subsection{Europe}
	\subsection{North America}
		\subsubsection{Canada}
			\paragraph{By Provinces and Territories}
			\paragraph{Rivers, Lakes, and Bodies of Water}
			
		\subsubsection{Mexico}
						\paragraph{By States and Territories}
						
			\paragraph{Rivers, Lakes, and Bodies of Water}
		\newpage
		\subsubsection{United States}
			\paragraph{By States and Territories} You should know the following states and territories. 
			


			\begin{tabular}{|c|c|c|c|}
				\hline
				\textbf{State} & \textbf{Capital} & \textbf{Nickname} & \textbf{Highest Point} \\
				\hline
				Alabama & Montgomery & The Yellowhammer State & Cheaha Mountain  \\
				\hline
				 Alaska & & & \\
				 \hline
				 Alaska 	& & &  \\
				 \hline
				 Arizona 	& & &  \\
				 \hline
				 Arkansas 	& & &  \\
				 \hline
				 California 	& & &  \\
				 \hline
				 Colorado 	& & &  \\
				 \hline
				 Connecticut 	& & &  \\
				 \hline
				 D.C. 	& & &  \\
				 \hline
				 Delaware 	& & &  \\
				 \hline
				 Florida 	& & &  \\
				 \hline
				 Georgia 	& & &  \\
				 \hline
				 Hawaii 	& & &  \\
				 \hline
				 Idaho 	& & &  \\
				 \hline
				 Illinois 	& & &  \\
				 \hline
				 Indiana 	& & &  \\
				 \hline
				 Iowa 	& & &  \\
				 \hline
				 Kansas 	& & &  \\
				 \hline
				 Kentucky 	& & &  \\
				 \hline
				 Louisiana 	& & &  \\
				 \hline
				 Maine 	& & &  \\
				 \hline
				 Maryland 	& & &  \\
				 \hline
				 Massachusetts 	& & &  \\
				 \hline
				 Michigan 	& & &  \\
				 \hline
				 Minnesota 	& & &  \\
				 \hline
				 Mississippi 	& & &  \\
				 \hline
				 Missouri 	& & &  \\
				 \hline
				 Montana 	& & &  \\
				 \hline
				 Nebraska 	& & &  \\
				 \hline
				 Nevada 	& & &  \\
				 \hline
				 New Hampshire 	& & &  \\
				 \hline
				 New Jersey 	& & &  \\
				 \hline
			
				 New Mexico 	& & &  \\
				 \hline
				 New York 	& & &  \\
				 \hline
				 North Carolina 	& & &  \\
				 \hline
				 North Dakota 	& & &  \\
				 \hline
				 Ohio 	& & &  \\
				 \hline
				 Oklahoma 	& & &  \\
				 \hline
				 	\end{tabular}
			 
		 \begin{tabular}{|c|c|c|c|}
				 \hline
				 Oregon 	& & &  \\
				 \hline
				 Pennsylvania 	& & &  \\
				 \hline
				 Rhode Island 	& & &  \\
				 \hline
				 South Carolina 	& & &  \\
				 \hline
				 South Dakota 	& & &  \\
				 \hline
				 Tennessee 	& & &  \\
				 \hline
				 Texas 	& Austin & The Lone Star State & Guadalupe Peak \\
				 \hline
				 Utah 	& & &  \\
				 \hline
				 Vermont 	& & &  \\
				 \hline
				 Virginia 	& & &  \\
				 \hline
				 Washington 	& & &  \\
				 \hline
				 West Virginia 	& & &  \\
				 \hline
				 Wisconsin 	& & &  \\
				 \hline
				 Wyoming 	& & &  \\
				 \hline
				 Puerto Rico	& & &  \\
				 \hline
				 US Virgin Islands	& & &  \\
				 \hline
				 Northern Mariana Islands	& & &  \\
				 \hline
				 Guam	& & &  \\
				 \hline
				 American Samoa	& & &  \\
				 \hline
				 District of Columbia	& & &  \\
				 \hline
				 Baker Island 	& & &  \\
				 \hline
				 Howland Island 	& & &  \\
				 \hline
				 Jarvis Island 	& & &  \\
				 \hline
				 Johnston Atoll 	& & &  \\
				 \hline
				 Kingman Reef 	& & &  \\
				 \hline
				 Midway Atoll 	& & &  \\
				 \hline
				 Navassa Island 	& & &  \\
				 \hline
				 Palmyra Atoll 	& & &  \\
				 \hline
				 Wake Island	& & &  \\
				 \hline
				 
				 
			\end{tabular}
		\newpage
		\paragraph{UNESCO World Heritage Sites:} UNESCO world heritage sites are landmarks or areas, selected by the United Nations Educational, Scientific and Cultural Organization (UNESCO) for having cultural, historical, scientific or other form of significance, which is legally protected by international treaties. The sites are judged to be important for the collective and preservative interests of humanity. In the United States, there are 11 cultural, 12 natural, and 1 mixed sites. 
		
		\begin{longtable}{|c|p{1.5 in}|p{4in}|}
			\hline
			\textbf{Type} & \textbf{Name and Date Designated} & \textbf{Description\footnote{Descriptions are summarized from UNESCO website, \href{https://whc.unesco.org/en/statesparties/us}{https://whc.unesco.org/en/statesparties/us}}} \\
			\hline
			Cultural & Cahokia Mounds State Historic Site (1982) (Illinois) & The largest pre-Columbian settlement north of Mexico.  Primary features at the site include Monks Mound, the largest prehistoric earthwork in the Americas, covering over 5 ha and standing 30 m high.\\
			\hline
			Cultural & Chaco Culture (1987) (New Mexico) & Chaco Canyon, a major center of ancestral Pueblo culture between 850 and 1250, was a focus for ceremonials, trade and political activity for the prehistoric Four Corners area. Chaco is remarkable for its monumental public and ceremonial buildings and its distinctive architecture \\
			\hline
			Cultural & Independence Hall (1979) (Phillidelphia, PA)& The Declaration of Independence (1776) and the Constitution of the United States (1787) were both signed in this building in Philadelphia. The universal principles of freedom and democracy set forth in these documents are of fundamental importance to American history and have also had a profound impact on law-makers around the world.\\
			\hline
			Cultural & La Fortaleza and San Juan National Historic Site in Puerto Rico (1983) & Between the 16th and 20th centuries, a series of defensive structures was built at this strategic point in the Caribbean Sea to protect the city and the Bay of San Juan. They represent a fine display of European military architecture adapted to harbour sites on the American continent.\\
			\hline
			Cultural &Mesa Verde National Park (1978) (Colorado) & A great concentration of ancestral Pueblo Indian dwellings, built from the 6th to the 12th century, can be found on the Mesa Verde plateau in south-west Colorado at an altitude of more than 2,600 m. Some 4,400 sites have been recorded, including villages built on the Mesa top. There are also imposing cliff dwellings, built of stone and comprising more than 100 rooms.\\
			\hline
			Cultural & Monticello and the University of Virginia in Charlottesville (1987) & Thomas Jefferson (1743–1826), author of the American Declaration of Independence and third president of the United States, was also a talented architect of neoclassical buildings. He designed Monticello (1769–1809), his plantation home, and his ideal 'academical village' (1817–26), which is still the heart of the University of Virginia. Jefferson's use of an architectural vocabulary based upon classical antiquity symbolizes both the aspirations of the new American republic as the inheritor of European tradition and the cultural experimentation that could be expected as the country matured. \\
			\hline
			Cultural & Monumental Earthworks of Poverty Point (2014) (Mississippi) & Monumental Earthworks of Poverty Point owes its name to a 19th-century plantation close to the site, which is in the Lower Mississippi Valley on a slightly elevated and narrow landform. The complex comprises five mounds, six concentric semi-elliptical ridges separated by shallow depressions and a central plaza. It was created and used for residential and ceremonial purposes by a society of hunter fisher-gatherers between 3700 and 3100 BP. It is a remarkable achievement in earthen construction in North America that was unsurpassed for at least 2,000 years.\\
			\hline
			Cultural & San Antonio Missions (2015) & The site encompasses a group of five frontier mission complexes situated along a stretch of the San Antonio River basin in southern Texas, as well as a ranch located 37 kilometres to the south. It includes architectural and archaeological structures, farmlands, residencies, churches and granaries, as well as water distribution systems. The complexes were built by Franciscan missionaries in the 18th century and illustrate the Spanish Crown’s efforts to colonize, evangelize and defend the northern frontier of New Spain. The San Antonio Missions are also an example of the interweaving of Spanish and Coahuiltecan cultures, illustrated by a variety of features, including the decorative elements of churches, which combine Catholic symbols with indigenous designs inspired by nature.\\
			\hline
			Cultural &Statue of Liberty (1984) (New York) & Made in Paris by the French sculptor Bartholdi, in collaboration with Gustave Eiffel (who was responsible for the steel framework), this towering monument to liberty was a gift from France on the centenary of American independence. Inaugurated in 1886, the sculpture stands at the entrance to New York Harbour and has welcomed millions of immigrants to the United States ever since.\\
			\hline
			Cultural &Taos Pueblo (1992) (New Mexico) & Situated in the valley of a small tributary of the Rio Grande, this adobe settlement – consisting of dwellings and ceremonial buildings – represents the culture of the Pueblo Indians of Arizona and New Mexico. \\
			\hline
			Cultural &The 20th-Century Architecture of Frank Lloyd Wright (2019) (Multiple Sites) & The property consists of eight buildings in the United States designed by the architect during the first half of the 20th century. These include well known designs such as Fallingwater (Mill Run, Pennsylvania) and the Guggenheim Museum (New York). All the buildings reflect the ‘organic architecture’ developed by Wright, which includes an open plan, a blurring of the boundaries between exterior and interior and the unprecedented use of materials such as steel and concrete. Each of these buildings offers innovative solutions to the needs for housing, worship, work or leisure. Wright's work from this period had a strong impact on the development of modern architecture in Europe.\\
			
			\hline
			Natural & Carlsbad Caverns National Park (1995) (New Mexico) & This karst landscape in the state of New Mexico comprises over 80 recognized caves. They are outstanding not only for their size but also for the profusion, diversity and beauty of their mineral formations. Lechuguilla Cave stands out from the others, providing an underground laboratory where geological and biological processes can be studied in a pristine setting. \\ 
			\hline
			Natural &Everglades National Park (1979) & This site at the southern tip of Florida has been called 'a river of grass flowing imperceptibly from the hinterland into the sea'. The exceptional variety of its water habitats has made it a sanctuary for a large number of birds and reptiles, as well as for threatened species such as the manatee.\\
			\hline
			Natural &Grand Canyon National Park (1979) & Carved out by the Colorado River, the Grand Canyon (nearly 1,500 m deep) is the most spectacular gorge in the world. Located in the state of Arizona, it cuts across the Grand Canyon National Park. Its horizontal strata retrace the geological history of the past 2 billion years. There are also prehistoric traces of human adaptation to a particularly harsh environment. \\
			\hline
			Natural &Great Smoky Mountains National Park (1983) & Stretching over more than 200,000 ha, this exceptionally beautiful park is home to more than 3,500 plant species, including almost as many trees (130 natural species) as in all of Europe. Many endangered animal species are also found there, including what is probably the greatest variety of salamanders in the world. Since the park is relatively untouched, it gives an idea of temperate flora before the influence of humankind.\\
			\hline
			Natural &Hawaii Volcanoes National Park (1987) & This site contains two of the most active volcanoes in the world, Mauna Loa (4,170 m high) and Kilauea (1,250 m high), both of which tower over the Pacific Ocean. Volcanic eruptions have created a constantly changing landscape, and the lava flows reveal surprising geological formations. Rare birds and endemic species can be found there, as well as forests of giant ferns.\\
			\hline
			Natural &Kluane / Wrangell-St. Elias / Glacier Bay / Tatshenshini-Alsek (1979,1992, 1994) &These parks comprise an impressive complex of glaciers and high peaks on both sides of the border between Canada (Yukon Territory and British Columbia) and the United States (Alaska). The spectacular natural landscapes are home to many grizzly bears, caribou and Dall's sheep. The site contains the largest non-polar icefield in the world.\\
			\hline
			Natural &Mammoth Cave National Park (1981) & Mammoth Cave National Park, located in the state of Kentucky, has the world's largest network of natural caves and underground passageways, which are characteristic examples of limestone formations. The park and its underground network of more than 560 surveyed km of passageways are home to a varied flora and fauna, including a number of endangered species.\\
			\hline
			Natural &Olympic National Park (1981) & Located in the north-west of Washington State, Olympic National Park is renowned for the diversity of its ecosystems. Glacier-clad peaks interspersed with extensive alpine meadows are surrounded by an extensive old growth forest, among which is the best example of intact and protected temperate rainforest in the Pacific Northwest. Eleven major river systems drain the Olympic mountains, offering some of the best habitat for anadromous fish species in the country. The park also includes 100 km of wilderness coastline, the longest undeveloped coast in the contiguous United States, and is rich in native and endemic animal and plant species, including critical populations of the endangered northern spotted owl, marbled murrelet and bull trout.\\
			\hline
			Natural &Redwood National and State Parks  (1980) &Redwood National Park comprises a region of coastal mountains bordering the Pacific Ocean north of San Francisco. It is covered with a magnificent forest of coastal redwood trees, the tallest and most impressive trees in the world. The marine and land life are equally remarkable, in particular the sea lions, the bald eagle and the endangered California brown pelican. \\
			\hline
			Natural &Waterton Glacier International Peace Park (1995) & In 1932 Waterton Lakes National Park (Alberta, Canada) was combined with the Glacier National Park (Montana, United States) to form the world's first International Peace Park. Situated on the border between the two countries and offering outstanding scenery, the park is exceptionally rich in plant and mammal species as well as prairie, forest, and alpine and glacial features. \\
			\hline
			Natural &Yellowstone National Park (1978) & The vast natural forest of Yellowstone National Park covers nearly 9,000 km2 ; 96\% of the park lies in Wyoming, 3\% in Montana and 1\% in Idaho. Yellowstone contains half of all the world's known geothermal features, with more than 10,000 examples. It also has the world's largest concentration of geysers (more than 300 geyers, or two thirds of all those on the planet). Established in 1872, Yellowstone is equally known for its wildlife, such as grizzly bears, wolves, bison and wapitis.\\
			\hline
			Natural &Yosemite National Park (1984) & Yosemite National Park lies in the heart of California. With its 'hanging' valleys, many waterfalls, cirque lakes, polished domes, moraines and U-shaped valleys, it provides an excellent overview of all kinds of granite relief fashioned by glaciation. At 600–4,000 m, a great variety of flora and fauna can also be found here.\\
			
			\hline
			Mixed & Papah\=anaumoku\={a}kea (2010) & a vast and isolated linear cluster of small, low lying islands and atolls, with their surrounding ocean, roughly 250 km to the northwest of the main Hawaiian Archipelago and extending over some 1931 km. The area has deep cosmological and traditional significance for living Native Hawaiian culture, as an ancestral environment, as an embodiment of the Hawaiian concept of kinship between people and the natural world, and as the place where it is believed that life originates and to where the spirits return after death. On two of the islands, Nihoa and Makumanamana, there are archaeological remains relating to pre-European settlement and use. Much of the monument is made up of pelagic and deepwater habitats, with notable features such as seamounts and submerged banks, extensive coral reefs and lagoons. It is one of the largest marine protected areas (MPAs) in the world. \\
			\hline
			
		\end{longtable}
			
			
			\newpage
			\paragraph{Rivers, Lakes, and Bodies of Water}
				\subparagraph{The Great Lakes} - There are 5 great lakes located along or near the US-Canada border.  At the end of the last Ice Age, there was a glacial lake, called Lake Agassiz that was larger than all of the lakes combined.  
				
				\begin{itemize}
					\item \textbf{Lake Superior} - Is the largest and deepest of all the great lakes.  It is mentioned in the Gordon Lightfoot song \textit{The Wreck of the Edmund Fitzgerald}.  The Chippewa called the lake \textit{Gitchigumi}.  Cities on Lake Superior include Duluth, MN, Thunder Bay, Ontario, and Sault Ste. Marie is located  between lakes Superior and Huron.  
					
					\item \textbf{Lake Michigan} - Chicago, Il, Green Bay, WI, and Gary, IN are located on lake Michigan.  It separates Michigan's upper peninsula from the rest of Michigan.  
					
					
					\item \textbf{Lake Huron}
					\item \textbf{Lake Erie}  Detroit, MI, Toledo OH, Cleveland OH, Erie PA, and Buffalo, NY are all located on the shores of lake Erie.  Niagara falls is on the Niagara River that connects Lake Erie to Lake Ontario.  
					
					\item \textbf{Lake Ontario}
				\end{itemize}
								
				\subparagraph{Other Lakes}
				\begin{itemize}
					
					\item{The Great Salt Lake}
					\item{Crater Lake}
					\item{Lake of the Woods}
					\item{Lake Okeechobee}
					\item{Lake Meade}
					\item{Lake Tahoe}
				\end{itemize}
				\subparagraph{Rivers}
		
			
	\subsection{South America}


	\section{Oceans}
		Though globally all oceans are connected and currents circulate water through all of the ocean, historically, there have been four named oceans: Atlantic, Arctic, Indian and Pacific. Recently, many countries, including the United States have began recognizing a fifth ocean: the Southern Ocean, which is the large ocean area that encircles Antarctica.  
		\subsubsection{Atlantic}
		\subsubsection{Arctic}
		\subsubsection{Indian}
		\subsubsection{Pacific}
		\subsubsection{Southern}
		
		
