\chapter{Current Events}

	\section{Introduction to Current Events}
		The very nature of current events makes it extremely hard to write a section of a book on this.  Studying for the current events means keeping up with the news.  Whether it be online news sites like \href{http://www.cnn.com}{CNN's website}, watching the news on TV or even reading the newspaper, you must do this every day to be well informed about current events.  
		
		This section will change quite frequently, and some of the current events section will undoubtedly be moved into the ``history'' section as time passes.  For reference, this version of \mytitle was released on \today. 
				
	\section{Business}
		\subsection{US-China Trade War}
		The US-China Trade war is an ongoing conflict between the two largest economies in the world: US and China.  In 2018, United States President Donald Trump announced that tariffs and other trade barriers would be placed on certain Chinese goods in response to what he called ``Unfair Trade Practices.''  In response, China, under the leadership of Xi Jinping retaliated with tariffs and other trade barriers would be placed on certain products imported from the United States.  
		
		\begin{itemize}
			\item Many farmers have struggled due to inability to export their crops to China.
			\item Manufacturers have had to pass on higher prices to consumers. 
			\item The trade war has strained relations between the United States and China.
			\item Stock Market volatility has followed announcements of tariffs and trade-talks.  
			\item Chinese telecommunication manufacturer Huawei received special attention in the trade war.  The United States warned its NATO allies that Huawei's equipment is believed to contain exploits that the Chinese Government may control.  

			
		\end{itemize}
		\subsection{Fed Policy}
		
		Beginning after the Great Recession of 2008, the federal reserve instituted a policy of \textit{quantitative easing} which consisted of the buying of government securities such as bonds in order to increase the money supply, and thus encourage lending and investment.  
		
		\subsection{Cryptocurrency and Blockchain}
		\def\bitcoinA{%
			\leavevmode
			\vtop{\offinterlineskip %\bfseries
				\setbox0=\hbox{B}%
				\setbox2=\hbox to\wd0{\hfil\hskip-.03em
					\vrule height .3ex width .15ex\hskip .08em
					\vrule height .3ex width .15ex\hfil}
				\vbox{\copy2\box0}\box2}}
		
		The first cryptocurrency, and with a market dominance of approximately 66\% is bitcoin.  Bitcoin was introduced in \textit{Bitcoin: A Peer-to-Peer Electronic Cash System}, nicknamed ``The White Paper'' by its creator, Satoshi Nakamoto in 2008.   Its symbol is \bitcoinA.  Satoshi Nakamoto is likely a pseudonym for a person or group of persons that created blockchain technology, which solves the ``double-spend'' problem for peer-to-peer networks.  
		\begin{itemize} 
			\item The first purchase made with cryptocurrency was two pizzas, sold for \bitcoinA10000 
			\item In late 2017, Bitcoin hit its all-time high exchange rate of nearly \$20000 for \bitcoinA1. 
			\item A deluge of ``Altcoins'' followed the creation of Bitcoin.  Some of the most successful are Etherium, Ripple, Litecoin, and Tether.  
			\item Due to competing ideals, bitcoin itself was hard-forked several times, with the alteratives calling themselves Bitcoin Cash, Bitcoin ABC, Bitcoin SV, and Bitcoin Gold.  
			\item In July of 2017, John David McAfee, the founder of McAfee antivirus software, made a famous prediction that bitcoin would reach \bitcoinA1 = \$500,000 in three years, or he would initiate his own creative punishment.  
			
		\end{itemize}
		\subsection{Gig Economy}
		\subsection{Retail Apocalypse}
		
		
	\section{Politics}
		\subsection{Scandals}
			\subsubsection{Scandals involving Donald Trump}
			\subsubsection{Scandals involving Hilary Clinton}
			\subsubsection{Scandals involving Jeffrey Epstein}
			\subsubsection{Scandals involving Joe Biden}
			\subsubsection{Scandals involving other US Politicians}

		\subsection{Foreign relations}	
	\section{Science}